\documentclass{article}
\usepackage{graphicx} % Required for inserting images
\usepackage{listings}
\usepackage{amsmath}

\title{TP PLP}
\author{Javaneta}


\begin{document}

\newcommand{\streq}[1]{\stackrel{\text{#1}}{=}}

\maketitle

\section{Ejercicio 9}
De acuerdo a las definiciones de las funciones para arboles ternarios (AT a) se pide demostrar: $(\forall$t::AT a)($\forall$x::a)(elem x (preorder t) = elem x (postorder t))  \\

Preorder y Postorder son dos maneras diferentes de recorrer un árbol; En este caso en puntual, recorren un árbol ternario.  \\

La lista resultante de ordenar el árbol ternario que devuelve preorder t y postorder t son diferentes, pero lo que ambos tienen en común es que contienen los mismos elementos, es decir, $(\forall x::a) (x \in listaResPreOrder \iff x \in listaResPostOrder)$ \\
Tenemos un caso que podemos mencionar, donde en el ejercicio 4 del TP: elem n (preorder at) = elem n (postorder at) es válido $\forall n::a$ \\

Por lo tanto, probemos esto utilizando los principios de extensionalidad e inducción estructural sobre árboles para concluir que esto es verdadero para cualquier árbol ternario.  \\

Recordemos la definición del tipo AT y cuáles son los constructores del tipo correspondientes: $ data \ AT \ a \ = \ Nil \ | \ Tern \ a \ (AT \ a) \ (AT \ a) \ (AT \ a) \ deriving \ Eq$ \\

Luego, recordemos qué ecuaciones representan a las operaciones de elem, preorder y postorder. 

\begin{lstlisting}
foldr :: (a -> b -> b) -> b -> [a] -> b
{FO0} foldr f z [] = z
{FO1} foldr f z (x:xs) = f x (foldr f z xs)

elem :: Eq a => a -> [a] -> Bool
{E0} elem e [] = False
{E1} elem e (x:xs) = (e==x) || elem e xs


foldAT :: (a -> b -> b -> b -> b) -> b -> AT a -> b 
{F0} foldAT _ b Nil = b 
{F1} foldAT f b (Tern r i c d)  = 
f r (foldAT f b i)  (foldAT f b c) (foldAT f b d)

preorder :: AT a -> [a]
{PR0} preorder = foldAT(\r i c d -> r : (i ++ c ++ d)) []

postorder :: AT a -> [a]
{PS0} postorder = foldAT(\r i c d -> reverse (r : (d ++ c ++ i))) []
\end{lstlisting}

Por inducción estructural en t tenemos: 
\begin{itemize}
    \item P(t) = ($\forall$ x::a)(elem x (preorder t) = elem x (postorder t))
\end{itemize}

Probemos el caso base, es decir, el constructor del tipo AT a no recursivo: este caso es Nil. \\ 
\textbf{Caso Base}: P(Nil) = ($\forall$ x::a)(elem x (preorder Nil) = elem x (postorder Nil)) \\ 
Resolvamos ambos lados por separado y deberíamos llegar a una equivalencia. 

\begin{itemize}
    \item elem x (preorder Nil) $\streq{PR0}$ elem x (foldAT(\r i c d $\rightarrow$ r : (i ++ c ++ d)) [] Nil) $\streq{F0}$ elem x [] $\streq{E0}$ False
    
    \item elem x (postorderNil) $\streq{PS0}$ elem x (foldAT(\r i c d $\rightarrow$ reverse (r : (d ++ c ++ i))) [] Nil) $\streq{F0}$ elem x [] = $\streq{E0}$ False
\end{itemize}

Por lo tanto, el caso base es verdadero. \\
Recordemos que para poder hacer inducción sobre listas necesitamos predicar acerca de $(\forall xs::[a])(\forall x ::a) (P(xs) \implies P(x:xs))$. En árboles, la inducción lo hacemos sobre cada constructor recursivo, en este caso serían 3. \\
\textbf{Paso Inductivo}: 
\begin{itemize}
    \item $(\forall i, c, d:: AT \ a)(\forall r::a)(P(i) \land P(c) \land P(d) \implies P(Tern \ r \ i \ c \ d))$
    \begin{itemize}
        \item HI: $P(i) \land P(c) \land P(d)$ donde: 
        \begin{itemize}
            \item P(i): elem x (preorder i) = elem x (postorder i)
            \item P(c): elem x (preorder c) = elem x (postorder c)
            \item P(d): elem x (preorder d) = elem x (postorder d)
        \end{itemize}
        \item TI: $P(Tern \ r \ i \ c \ d)$ es decir
        \begin{itemize}
        \item elem x (preorder $(Tern \ r \ i \ c \ d)$) = elem x (postorder $(Tern \ r \ i \ c \ d)$)
        \end{itemize}
    \end{itemize}
\end{itemize}
Por lo tanto \\ 
elem x (preorder (Tern r i c d)) \\ 
$\streq{PR0}$ elem x (foldAT($\backslash$r i c d $\rightarrow$ r : (i ++ c ++ d)) [] (Tern r i c d)) \\ 
$\streq{F1}$ elem x (r : (foldAT ($\backslash$r i c d $\rightarrow$ r : (i ++ c ++ d)) [] i ++ foldAT ($\backslash$r i c d $\rightarrow$ r : (i ++ c ++ d)) [] c ++ foldAT ($\backslash$r i c d $\rightarrow$ r : (i ++ c ++ d)) [] d)) \\ 
$\streq{PR0}$ elem x (r : (preorder i) ++ (preorder c) ++ (preorder d)) \\
$\streq{E1}$ (x==r) $\|$ elem x ((preorder i) ++ (preorder c) ++ (preorder d)) \\
Aplicamos el Lema 1 \\
= (x==r) $\|$ elem x (preorder i) $\|$ elem x (preorder c) $\|$ elem x (preorder d) \\
Por extensionalidad de booleanos en la condición de (x==r) tenemos dos casos:
\begin{itemize}
\item A: (x==r) = True 
\item B: (x==r) = False
\end{itemize}
Caso A: True $\|$ elem x (preorder i) $\|$ elem x (preorder c) $\|$ elem x (preorder d) $\rightarrow$, esto es verdadero ya que en el caso de un ó lógico basta un True para que todo sea verdadero.  \\     
Caso B: False $\|$ elem x (preorder i) $\|$ elem x (preorder c) $\|$ elem x (preorder d) $\streq{HI}$ elem x (postorder i) $\|$ elem x (postorder c)  $\|$ elem x (postorder d) $\streq{Lema 1}$ elem x ((postorder(i)) ++ (postorder c) ++ (postorder d) \\

\textbf{Lema 1}: Vamos a probar la siguiente propiedad sobre inducción estructural sobre xs. \\
$(\forall xs :: [a])(\forall ys :: [a])(\forall e::a)(elem \ e \ (xs++ys) \ = elem \ e \ xs \ \| \ elem \ e \ ys$ \\
P(xs) = $(\forall ys :: [a])(\forall e::a)(elem \ e \ (xs++ys) \ = elem \ e \ xs \ \| \ elem \ e \ ys$ \\
\textbf{Caso Base}: P([])
\begin{itemize}
    \item elem e ([] ++ ys) = elem e (foldr(:) ys []) $\streq{F0}$ elem e ys
    \item elem e ([]) $\|$ elem e (ys) $\streq{E0}$ False $\|$ elem e ys = elem e ys
\end{itemize}
Por lo tanto, el caso base es verdadero. \\
\textbf{Paso Inductivo}: 
\begin{itemize}
    \item HI: P(xs): $(\forall ys :: [a])(\forall e::a)(elem \ e \ (xs++ys) \ = elem \ e \ xs \ \| \ elem \ e \ ys$
    \item TI: P(x:xs): $(\forall ys :: [a])(\forall e::a)(elem \ e \ ((x:xs)++ys) \ = elem \ e \ (x:xs) \ \| \ elem \ e \ ys$
\end{itemize}
elem e ((x:xs) ++ ys) $\streq{++}$ elem e (foldr (:) ys (x:xs)) $\streq{FO1}$ elem e ((:) x (foldr(:) ys xs) $\streq{E1}$ (e == x $\|$ elem e (foldr(:) ys xs) \\
Por el \textbf{principio de extensionalidad de booleanos} tengo dos casos para $e==x$
\begin{itemize}
    \item LE1) $e==x = True$ = True $\|$ elem e xs $\|$ elem e ys $=$ True
    \item LE2) $e==x = False$ = False $\|$ elem e xs $\|$ elem e ys $=$ elem e xs $\|$ elem e ys 
\end{itemize}
El caso LE1 es trivial. El caso LE2 podemos aplicar la HI \\
elem e xs $\|$ elem e ys $\streq{HI}$ elem e (xs ++ ys)


\end{document}

